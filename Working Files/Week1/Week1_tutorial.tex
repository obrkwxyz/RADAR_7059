\documentclass[a4paper,12pt]{article} % The document class with options
\usepackage[T1]{fontenc}
\usepackage{amsmath, amsfonts, amssymb}
\usepackage{graphicx}
\usepackage[margin=2cm]{geometry}
\usepackage[many]{tcolorbox}
\usepackage[shortlabels]{enumitem}
\setlength{\parindent}{0em}
\newtcolorbox{mybox}[1]{%
    tikznode boxed title,
    enhanced,
    arc=0mm,
    interior style={white},
    attach boxed title to top center= {yshift=-\tcboxedtitleheight/2},
    fonttitle=\bfseries,
    colbacktitle=white,coltitle=black,
    boxed title style={size=normal,colframe=white,boxrule=0pt},
    title={#1}}

\title{Week 1 Tutorial}
\author{Kane O'Brien}
\date{\today}
\begin{document}
\maketitle

\textbf{Question 1:} Convert to dBW and dBm the following power levels:
\begin{enumerate}[(a)]
    \item $100W$
    
        $$10 \log(100) = 20 dBW $$
        $$10 \log(100 \times 10^3) = 50 dBm$$ 
    \item $10kW$
    
        $$10 \log(10 \times 10^3) = 40 dBW$$
        $$10 \log(100 \times 10^3\times 10^3) = 70 dBm$$
    \item $0.1W$
    
        $$10 \log(0.1) = -10 dBW$$
        $$10 \log(0.1 \times 10^3) = 20 dBm$$

    \item $10^-14$W
    
        $$10 \log(10^{-14}) = -140 dBW$$
        $$10 \log(10^{-11}) = -110 dBm$$

\end{enumerate}

\textbf{Question 2:} Find the delay times associated with the following target ranges:

\vspace{1ex}
$$ v =\frac{d}{t} => t=\frac{d}{v} => t=\frac{2R}{v}$$
\begin{enumerate}[(a)]
    \item $5$m
    
    $$t=\frac{2.5}{3\times10^8} = \frac{10}{3\times10^8} = 33 nS$$
    \item $1$ km
    
    $$t=\frac{2\times1\times10^3}{3\times10^8} = \frac{2\times10^3}{3\times10^8} = 6.67 \mu S$$
    \item $100$ km
    
    $$t=\frac{2\times100\times10^3}{3\times10^8} = \frac{200\times10^3}{3\times10^8} = 666.67 \mu S$$
    \item $3000$ km
    
   $$t=\frac{2\times3000\times10^3}{3\times10^8} = \frac{6000\times10^3}{3\times10^8} = 20mS$$
\end{enumerate}

\vspace{1ex}
\textbf{Question 3:} The intensity of a transmitted electromagnetic wave at a range of 500m from the radar is $0.04$ W/m$^2$. What is the intensity at 2 km?

\vspace*{1ex}
Range increases by 3x, $U_0 \propto \frac{1}{R^2}$; therefore the power is proportional to $\frac{1}{3^2}$
\vspace*{1ex}

$$\frac{1}{9}\times 0.04 = 4.44 mW/m^2 $$

\vspace{1ex}
\textbf{Question 4:} Consider a 1 MW transmitter through a 30 dB antenna. What would the spatial power density on the surface of the moon assuming no losses, free space propagation etc? Use 384000km for the distance.

\vspace*{1ex}
$$P_r = \frac{P_t G_t}{4\pi R^2} \text{    where $G_t$, linear = $10^3$}$$ 

$$P_r = \frac{1\times10^6 \times10^3}{4\pi \times384000^2}$$

\vspace*{1ex}
$$P_r = 539.6 \mu W/m^2$$
$$P_r = -32 dBW $$


\vspace{1ex}
\textbf{Question 5:} A radar transmits rectangular pulses of width $1 \mu$s and peak power 1MW. Determine the range resolution and energy of the transmitted pulse.

\vspace*{1ex}
Range Resolution : $\tau = \frac{2R}{c} => R = \frac{\tau c}{2}$

$$ R = \frac{\tau c}{2} = \frac{1\times10^{-6} \times 3\times10^8}{2} = \frac{3^2}{2} = 4.5m $$

\vspace*{1ex}
Pulse energy : Integrate the pulse width over time. $\int P_t d\tau$

$$ = 1\times10^6\times1\times10^{-6} = 1W$$


\vspace{1ex}
\textbf{Question 6:} What (uncompressed) pulse width is required to give a range resoution of 15 m?

\vspace*{1ex}
$$\tau = \frac{2R}{c} = \frac{2\times 15}{3\times10^8} = \frac{30}{3\times10^8} = 100 nS$$


\vspace{2ex}
\textbf{Question 7:}
 When an input voltage of 2V is applied to an amplifier an output voltage of 64V is measured. What is the gain of the amplifer in linear and logarithmic units?

 \vspace*{1ex}

 $$Av = \frac{V_{out}}{V_{in}} = \frac{64}{2} = 32x $$ linear

 $$Av_{(dB)} = 20 \log (32) = 30.102 dB$$

\end{document}